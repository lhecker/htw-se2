\part{Projektdokumentation}
\todo{kleine Einleitung}
\todo{Seitennummerierung zurücksetzen!}
\chapter{Anforderungs- und Problemanalyse}
\section{Methoden}
Aufgabe der Anforderungsanalyse in diesem Projekt war es herauszufinden
welches Problem der Kunde, in unserem konkreten Fall unsere Professorin Frau 
Hauptmann mit der zu entwickelnden Software lösen möchte. Dafür wurden 
Interviews mit dem Kunden durchgeführt und entsprechende Ergebnisse mit Hilfe 
von Audioaufzeichnung, Mitschrift und Fotografien protokolliert. Zur 
detaillierten Beschreibung einzelner Abläufe des Systems wurden 
Kreativtechniken wie das Zeichnen verschiedener Szenarien an einem Whiteboard 
sowie die manuelle Simulation des Fahrstuhles mit einem aus Pappe gefertigten Modells durchgeführt.
\begin{figure}[hbt]
\centering
\subcaptionbox{Skizze der Fahrstuhlsimulation am Whiteboard}[0.49\linewidth]
{\includegraphics[height=8cm]{images/kundengespraech1.jpg}}
\subcaptionbox{Modell des Fahrstuhles}[0.49\linewidth]
{\includegraphics[height=8cm]{images/pappfahrstuhl.jpg}}
\caption{Kreativtechniken zur Anforderungsanalyse}
\end{figure}
Grundlegende Fragen die im Laufe der Analyse geklärt werden mussten waren
\begin{itemize}
	\item Wie viele Fahrstühle sollen verwendet werden können?
	\item Wie soll das Gebäude beschaffen sein?
	\item Welcher Algorithmus soll verwendet werden?
	\item Gibt es Schnittstellen zu anderen Systemen?
\end{itemize}
Weiterhin musste festgelegt werden ob die Priorität des Systems auf der 
Simulation oder auf einer möglichst realitätsnahen Umsetzung eines Liftes liegt.
Im Laufe der Analyse und Modellierung entsprechender Anwendungsfälle 
wurde ersichtlich, dass das System sich aus zwei Teilsystemen, der 
\textbf{Fahrstuhlsteuerung} und der \textbf{Fahrstuhlsimulation} zusammensetzt, 
deren Anforderungen getrennt voneinander beschrieben werden mussten.\\
Eine Besonderheit des Systems ist die Umgebung in der es eingesetzt werden 
soll, der Lehrbetrieb an einer Hochschule. Daraus ergaben sich spezielle 
Anforderungen wie das Anzeigen der Zustandsübergänge die gesondert betrachtet 
werden mussten.
\section{Diagramme}
\subsubsection{Zustandsdiagramm}
Die wesentliche Funktionalität sowie der Algorithmus des Systems lassen sich in 
dem Zustandsdiagramm abbilden. Um das bestmögliche Ergebnis zu erhalten haben 
wir verschiedene Versionen des Diagramms entworfen und diese in Gruppentreffen diskutiert und überarbeitet.
\missingfigure[figwidth=\textwidth]{Zustandsdiagramm ggf. mehrere Versionen}


\chapter{Software-Entwurf}
Für die technische Umsetzung der Zustände in ausführbaren Quellcode ergaben sich folgende Entwurfsmuster:
\todo{Wir müssen aufpassen, dass wir eine klare Trennung zwischen Entwurfs-spezifischen Inhalten im Entwicklerhandbuch und hier vornehmen.}
\subsection*{State Design Pattern}
Vorteile, Nachteile, warum haben wir uns dageben entschieden?
\subsection*{Zustandstabelle}
\subsection*{Event Methode}
ist das eine korrekte Bezeichnung? Falls nein, wie heißt das bei uns?


\chapter{Qualitätssicherung}


\chapter{Software-Test}


\chapter{Team-Organisation}
\section{Gruppenarbeit}
Um das Zusammenarbeiten in der Gruppe einfacher zu gestalten wurden verschiedene Technologien eingesetzt.
Zu finden von Terminen für Meetings wurde Doodle\footnote{\url{www.doodle.com}}  verwendet. Für die Kommunikation in der Gruppe Slack\footnote{\url{https://slack.com/}}.
\todo{Hier muss hin warum wir uns für JS und Co entschieden haben...}
\section{Verwendete Werkzeuge}
\section{Resümee}