\documentclass[fontsize=11pt,paper=a4,titlepage,DIV=calc,draft=false]{scrreprt}
% 11pt: Normale Textkörpergröße
% a4paper: Größe des Druckmediums
% titlepage: Titel auf einer separaten Seite ohne Seitenzahl
% twoside: Zweiseitiges Layout
% openright: Kapitel beginnen immer auf der rechten Seite
% headsepline: Trennt Textkörper von Headings durch Strich (entspr.: /footsepline)
% headinclude,footinclude: Kopf- und Fußzeile zählen zum Textkörper
% DIV=calc: Für die gewählten Optionen wird ein optimales Seitenverhältnis errechnet
% draft=true: Für Bilder wird die Box freigehalten, erheblicher Geschwindigkeitsvorteil.
% abstract: Setzt den Titel 'Zusammenfassung' vor den abstract

\usepackage{hyperref}
\usepackage{nameref}

\usepackage{testcase}

\usepackage{upgreek}
%\usepackage{subfigure}

%  %  %  %  Bindungskorrektour  %  %  %  %
% \KOMAoptions{BCOR=10mm}

%  %  %  %  Abkürzungen  %  %  %  %
% Das Einführen dieser Befehler verhindert Umbrüche bei mehrgliedrigen Abkürzungen
\usepackage{xspace}
\newcommand{\zB}{\mbox{z.\,B.}\xspace}
% Abkürzung für zum Beispiel


%  %  %  %  Einheiten  %  %  %  %
%\usepackage[thinspace,thinqspace,squaren,textstyle]{SIunits}
% Komfoatables Paket zum Einbinden von Einheiten


%  %  %  %  Kodierung, Schrift und Sprache  %  %  %  %
\usepackage[utf8]{inputenc}
\usepackage{palatino}
\usepackage[ngerman]{babel}
% damit man Text aus dem PDF korrekt rauskopieren kann


%  %  %  %  Grafiken, Tabellen, Mathematikumgebungen  %  %  %  %
\usepackage{graphicx}
\usepackage{tabularx}
\usepackage{xcolor}
\definecolor{halfgray}{gray}{0.55}
\usepackage{amsmath,amsfonts,amssymb}
\usepackage{flafter,afterpage}
\usepackage[section]{placeins}
\usepackage{setspace} \onehalfspacing
\usepackage[margin=8mm,font=small,labelfont=bf,format=plain]{caption}
\usepackage[margin=8mm,font=small,labelfont=bf,format=plain]{subcaption}

\numberwithin{equation}{chapter}
\numberwithin{figure}{chapter}
\numberwithin{table}{chapter}


%  %  %  %  Kopf- und Fußzeilen  %  %  %  %

% \renewcommand\frontmatter{\pagenumbering{Roman}}
\usepackage{chngcntr}
\counterwithout{footnote}{chapter}

% Zeilenabstand zwischen zwei Fußnoten:
\footnotesep 9pt
% Einrücken der Fußnoten:
\deffootnote[1.5em]{1em}{1.5em}{\thefootnotemark\ \ }
%
%\usepackage{fancyhdr}				% Paket für leicht konfigurierbare Kopf- und Fußzeilen
%\fancypagestyle{plain}{				% Neue Gestaltung der Chapter- Page
%\fancyhf{} 							% Clear all header and footer fields
%\renewcommand{\headrulewidth}{0pt}		% Keine Trennlinie zwischen Kopf- / Fußzeile und Textkörper
%\renewcommand{\footrulewidth}{0pt}}
%
%\fancypagestyle{myfoot}{				% Neue Gestaltung der frontmatter pages
%\fancyhf{}							% Clear all header and footer fields
%\fancyhead[RO]{\thepage}				% Seitenzahl außen auf ungeraden Seiten
%\fancyhead[LE]{\thepage}				% Seitenzahl außen auf geraden Seiten
%\renewcommand{\headrulerwidth}{0pt}	% Keine Trennlinie zwischen Kopf- / Fußzeile und Textkörper
%\renewcommand{\footrulerwidth}{0pt}}
%
%\pagestyle{fancy}					% Pagestyle fancy aktiviert selbstkonfigurierten Style
%\fancyhf{} 							% Alle Kopf- und Fußzeilenfelder werden zunächst bereinig
%\renewcommand{\headrulewidth}{0pt}		% Keine Trennlienie zwischen Kopfzeile und Textkörper
%
%%\renewcommand{\chaptermark}[1]{\markboth{#1}{}}
%%\renewcommand{\sectionmark}[1]{\markright{#1}{}}
%
%\fancyhead[RO]{\leftmark ~~~~ \thepage}
%\fancyhead[LE]{\thepage ~~~~ \nouppercase \rightmark}


%  %  %  %  Überschriften  %  %  %  %


%  %  %  %  Verzeichnisse  %  %  %  %

% % % Literaturverzeichnis % % %
%\usepackage{natbib}

% % % Inhaltsverzeichnis % % %
% Die Chaptereinträge:
\usepackage{titletoc}

\titlecontents{chapter}
	[0pc]
	{
		\addvspace{0.5pc}
		%\filouter}
	}
	{\sffamily\Large\thecontentslabel\quad\sffamily\Large}{}
	{\titlerule*[0.75pc]{}\enskip\rmfamily\Large \contentspage}  % Wäre mit Seitenzahl 																					rechtsbündig
	[\addvspace{.5pc}]

% Die Sectioneinträge:
\titlecontents{section}
	[3.78em]
	{}
	{\rmfamily\contentslabel{2.3em}\rmfamily}
	{\hspace*{-2.3em}}
	{\titlerule*[0.75pc]{.}\enskip\contentspage}
	[\addvspace{.1em}]

% Die Subsectioneinträge:
\titlecontents{subsection}
	[6.2em]
	{}
	{\rmfamily\contentslabel{2.3em}\rmfamily}
	{\hspace*{-2.3em}}
	{\titlerule*[0.75pc]{.}\enskip\contentspage}
	[\addvspace{.1em}]

%\titlecontents{subsection}
	%[6.8em]
	%{}
	%{\rmfamily\normalsize\contentslabel{3em}\rmfamily\large}
	%{\hspace*{-2.3em}}
	%{\titlerule*[0.75pc]{.}\enskip\contentspage}

% Glossar einbinden und Abkürziungsverzeichnis erstellen
\usepackage[acronym,toc,section=section,numberedsection,nonumberlist]{glossaries}
\loadglsentries{./glossar}
\makeglossaries
\makeindex

% Zähler für Parts zurücksetzen
\makeatletter
\@addtoreset{chapter}{part}
\makeatother

\usepackage{todonotes}
\usepackage{listings}
\lstset{basicstyle=\ttfamily\normalsize,breaklines=true}

% Javascript Support
\usepackage{color}
\definecolor{lightgray}{rgb}{.9,.9,.9}
\definecolor{darkgray}{rgb}{.4,.4,.4}
\definecolor{purple}{rgb}{0.65, 0.12, 0.82}
\lstdefinelanguage{JavaScript}{
  keywords={break, case, catch, continue, debugger, default, delete, do, else, false, finally, for, function, if, in, instanceof, new, null, return, switch, this, throw, true, try, typeof, var, void, while, with},
  morecomment=[l]{//},
  morecomment=[s]{/*}{*/},
  morestring=[b]',
  morestring=[b]",
  ndkeywords={class, export, boolean, throw, implements, import, this},
  keywordstyle=\color{blue}\bfseries,
  ndkeywordstyle=\color{darkgray}\bfseries,
  identifierstyle=\color{black},
  commentstyle=\color{purple}\ttfamily,
  stringstyle=\color{red}\ttfamily,
  sensitive=true
}

\lstset{
   language=JavaScript,
   backgroundcolor=\color{lightgray},
   extendedchars=true,
   basicstyle=\footnotesize\ttfamily,
   showstringspaces=false,
   showspaces=false,
   numbers=left,
   numberstyle=\footnotesize,
   numbersep=9pt,
   tabsize=2,
   breaklines=true,
   showtabs=false,
   captionpos=b
}

\makeatletter
\def\testclr#1#{\@testclr{#1}}
\def\@testclr#1#2{{\fboxsep\z@\fbox{\colorbox#1{#2}{\phantom{XX}}}}}
\newcommand*\Color[1]{\textsl{#1}}
\makeatother
% Define colors uses in application and images
\definecolor{rot}{HTML}{F5B1B1} 
\definecolor{blau}{HTML}{C5CEF3} 
\definecolor{gelb}{HTML}{F6EFDE} 
\definecolor{state-active}{HTML}{5CB85C} 
\definecolor{state-last}{HTML}{F0AD4E} 
\definecolor{state-lastlast}{HTML}{D9534F} 