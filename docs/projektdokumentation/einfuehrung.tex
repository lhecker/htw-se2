\chapter{Einführung}
\paragraph{}
Im Rahmen der Belegarbeit im Modul \textsc{Software Engineering 2} ist ein Software-System zu implementieren, welches die Steuerung eines Fahrstuhls simuliert. Dieses Software-System soll in der Zukunft als Anschauungsmaterial im Lehrbetrieb verwendet werden. Studierenden soll damit ermöglicht werden, die Zusammenhänge zwischen real existierenden Automaten und der Thematik der Zustandsdiagramme zu erfahren.
\\
Die Dokumentation des Projektes gliedert sich in folgende Teildokumentationen:
\subsubsection*{Benutzerhandbuch}
Im Benutzerhandbuch werden Anweisungen für die korrekte Verwendung der Software gegeben. Sie kann Mitarbeitern oder Studierenden, die die Anwendung verwenden möchten, zur Verfügung gestellt werden. Neben den Hinweisen zur Verwendung sind die Systemvorraussetzungen sowie Installationsanweisungen enthalten.

\subsubsection*{Entwicklerhandbuch}
Damit es möglich ist die Anwendung weiterzuentwickeln werden im Entwicklerhandbuch die internen Zusammenhänge und Strukturen dokumentiert. Enthalten sind die Klassendiagramme, sowie die Auflistung der Funktionen der \gls{API}.\\

\subsubsection*{Projektdokumentation}
Der Hauptteil der Dokumentation enthält die beim Projekt verwendeten Arbeitsschritte, verwendete Werkzeuge und Begründungen warum bestimmte Entscheidungen getroffen wurden.
Hier wird auch auf die Themen Analyse und Test eingegangen.
\\\\
Die einzelnen Teile der Dokumentationen sind in sich abgeschlossen und können unabhängig voneinander verwendet werden.

\section*{Konventionen}
Folgende Konventionen werden im Dokument verwendet:\\
\todo{ausformulieren}
Quellcode in Monospace\\
Innerhalb des Quellcodes camelNotation u.s.w.