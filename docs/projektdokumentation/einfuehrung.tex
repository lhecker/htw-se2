\chapter{Einführung}
\paragraph{}
Im Rahmen der Belegarbeit im Modul \textsc{Software Engineering 2} war ein Software-System zu implementieren, welches die Steuerung eines Fahrstuhls simuliert. Dieses Software-System soll in der Zukunft als Anschauungsmaterial im Lehrbetrieb verwendet werden. Studierenden soll damit ermöglicht werden, die Zusammenhänge zwischen real existierenden Automaten und der Thematik der Zustandsdiagramme zu erfahren.
\\
Die Dokumentation des Projektes gliedert sich in folgende Teildokumentationen:
\subsubsection*{Benutzerhandbuch}
Im Benutzerhandbuch werden Anweisungen für die korrekte Verwendung der Software gegeben. Sie kann Mitarbeitern oder Studierenden zur Verfügung gestellt werden, welche die Anwendung verwenden möchten. Neben den Hinweisen zur Verwendung sind die Systemvoraussetzungen sowie Installationsanweisungen enthalten.

\subsubsection*{Entwicklerhandbuch}
Um Eine Weiterentwicklung der Anwendung zu ermöglichen werden im Entwicklerhandbuch die internen Zusammenhänge und Strukturen dokumentiert. Enthalten sind die Klassendiagramme, sowie die Auflistung der Funktionen der \acrshort{API}.\\

\subsubsection*{Projektdokumentation}
Dieser Teil der Dokumentation wird sich mit Organisation der Projektarbeit beschäftigen. Analysiert werden die Herangehensweisen, verwendete Werkzeuge und verschiedene Entscheidungen die während der Projektarbeit getroffen worden. Ziel ist es die Zusammenarbeit und die Projektrealisierung zu reflektieren und entsprechende Schlüsse zu ziehen.

Ebenfalls behandelt dieser Abschnitt Themen der Analyse, der Qualitäts"-sicherung und des Software-Test.

\subsection*{Konventionen}
Folgende Konventionen werden im Dokument verwendet:\\
\begin{itemize}
	\item das entwickelte Software-System wird im folgenden Fahrstuhlsimulation genannt.
	\item Ordnernamen und Pfadangaben, sowie Codeausschnitte im laufenden Text sind durch eine nichtproportionale Schriftart gekennzeichnet
	\item Abkürzung werden nur bei der ersten Verwendung näher beschrieben, danach können sie zusätzlich im Glossar nachgeschlagen werden.
	\item Als Auftraggeberin wirkte Frau Professor Hauptmann, im folgenden als Kundin bezeichnet.
\end{itemize}
