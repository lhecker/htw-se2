\part{Benutzerhandbuch}
\chapter{Systemvoraussetzungen und Installation}
Die Anwendung \textbf{Fahrstuhlsimulation} liegt in mehreren Varianten vor. Die Nutzung von verschiedenen Webtechnologien ermöglichte es die Anwendung sowohl als ausführbare Datei für \textit{MAC OS X}, \textit{Linux} und \textit{Windows} als auch als statische HTML-Datei zur Verfügung zu stellen. Eine gesicherte und getestete Funktionalität bieten die folgenden Plattformen:

\begin{itemize}
	\item \textbf{Betriebssysteme:}
	\begin{itemize}
		\item MAC OS X ab Version 10.9
		\item Linux ab Version 3.0
		\item Microsoft Windows 7 und 8
	\end{itemize}
	\item \textbf{Browser:}
	\begin{itemize}
		\item Firefox Version 38 (Linux x64)
		\item Google Chrome Version 43
		\item Safari Version 8
		\item Internet Explorer 11 (Windows 7)\todo{automatischer Test der statischen Website}
	\end{itemize}
\end{itemize}

Um die Anwendung benutzen können ist ausschließlich notwendig die korrekte Version für die jeweilige Plattform zu kopieren und mit einem Doppelklick zu starten. Sollte sie das genutzte System nicht unter den getesteten befinden kann die Anwendung auch durch Doppelklick auf die statische HTML-Datei gestartet werden. In diesem Fall würde der systemeigene Browser zur Ausführung der Anwendung genutzt werden.

\chapter{Anwendung}
Das Anwendungsfenster ist in drei Teilbereiche unterteilt. Links ist die Ge"-bäude"-ansicht zu sehen, welche den gesamten Fahrstuhlschacht inklusive Fahrstuhl repräsentiert. Die Fahrten des Fahrstuhles werden durch das Bewegen der Fahrstuhlkabine im Fahrstuhlschacht und durch das Öffnen und Schließen der Türen in den entsprechenden Etagen visualisiert. Um in einer bestimmten Etage einen Fahrstuhlruf abzugeben, ist auf die entsprechenden Pfeiltasten der Etage zu klicken. Diese beiden Tasten ermöglichen es einen Ruf entweder \textit{nach oben} oder \textit{nach unten} abzugeben.

\paragraph{}Im oberen Teil der Anwendung ist das Tableau zur Etagenwahl innerhalb der Fahrstuhlkabine dargestellt. Hier können, vorausgesetzt es befinden sich Passagiere im Fahrstuhl, Fahrtwünsche abgegeben werden. Ebenfalls in diesem Bereich befindet sich eine weitere Anzeige der aktuellen Etage und das Symbol zur Anzeige einer Überlastsituation.\todo{Überprpfung der Erklärung mit finaler GUI} 

\paragraph{}Im rechten Bereich der Anwendung ist das Zustandsdiagramm des Systems dargestellt. Neben dem Subautomat des Türsystems zeigt es deutlich die beiden Stränge für die Fahrtrichtungen nach oben und nach unten.

\paragraph{}Durch Interaktion mit dem Fahrstuhlsystem ausgelöste Vorgänge und Zustandsübergänge werden in diesem Bereich durch farbliche Veränderungen ansprechend dargestellt. 

\missingfigure[figwidth=\textwidth]{Screenshot der Anwendung mit Farbig hinterlegten Teilen}
\todo{Screenshot mit Erläuterungen}
