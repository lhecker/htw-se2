\part{Benutzerhandbuch}

\chapter{Systemvoraussetzungen}
Da die Anwendung mit Hilfe von Webtechnologien realisiert wurde, ist sie Plattform-unabhängig und setzt lediglich einen modernen Browser \todo{was heißt modern wie siehts mit IE aus, welche Version} voraus. Getestet wurde sie unter
\begin{itemize}
 \item Firefox Version 38 (Linux x64)
 \item Google Chrome Version 
 \item Safari Version \todo{Mac Version ergänzen}
 \item Internet Explorer 11 (Windows 7) 
\end{itemize}
\chapter{Installation}
\chapter{Anwendung}
Das Anwendungsfenster ist in drei Teilbereiche unterteilt. Links ist die Gebäude\-ansicht zu sehen, die den gesamten Fahrstuhlschacht repräsentiert. Die Fahrten des Fahrstuhles werden durch das Bewegen der Fahrstuhlkabine im Fahrstuhlschacht und das Öffnen der Türen in den entsprechenden Etagen simuliert.
Im oberen Teil der Anwendung ist das Tableau zur Wahl der Etage innerhalb der Fahrstuhlkabine dargestellt. Hier kann, vorausgesetzt es befinden sich ein Passagier im Fahrstuhl, die Zieletage gewählt werden.

\missingfigure[figwidth=\textwidth]{Screenshot der Anwendung mit Farbig hinterlegten Teilen}
\todo{Ich würde, wenn die GUI fertig ist noch einen Screenshot mit Nummern und entsprechender Erklärung hinzufügen (ähnlich einer Bedienungsanleitung)}
