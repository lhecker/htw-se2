\chapter{Einführung}
\paragraph{}
Im Rahmen der Belegarbeit im Modul \textsc{Software Engineering 2} ist ein Software-System zu implementieren, welches die Steuerung eines Fahrstuhls simuliert. Dieses Software-System soll in der Zukunft als Anschauungsmaterial im Lehrbetrieb verwendet werden. Studierenden soll damit ermöglicht werden, die Zusammenhänge zwischen real existierenden Automaten und der Thematik der Zustandsdiagramme zu erfahren.

\paragraph{}
In diesem Zusammenhang ergeben sich zusätzlich zu den Anforderungen an das Teilsystem \textit{\textbf{\gls{Fahrstuhlsteuerung}}}, spezielle Anforderungen an das Teilsystem \textit{\textbf{\gls{Visualisierung}}} aus der Sicht des Lehrbetriebes. Im folgenden werden diese beiden Teilsysteme daher an verschiedenen Stellen getrennt voneinander betrachtet und beschrieben.

\paragraph{}
Das vorliegende Pflichtenheft dient der Beschreibung und Vereinbarung von Anforderungen an das Gesamtsystem \textit{\textbf{\gls{Fahrstuhlsimulation}}} und besitzt Vertragscharackter. Die darin enthaltenen Anforderungen wurden auf Basis von Kundengesprächen und Kundenvereinbarungen sowie der Aufgabenstellung formuliert\footnote{Meeting Minutes und Audiomitschnitte der beiden Kundengespräche sind unter folgendem Link zu finden: \url{http://goo.gl/UVHn2G}} . 

