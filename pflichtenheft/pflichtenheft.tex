% !TEX encoding = UTF-8 Unicode
\documentclass[fontsize=11pt,paper=a4,titlepage,DIV=calc,draft=false]{scrreprt}
% 11pt: Normale Textkörpergröße
% a4paper: Größe des Druckmediums
% titlepage: Titel auf einer separaten Seite ohne Seitenzahl
% twoside: Zweiseitiges Layout
% openright: Kapitel beginnen immer auf der rechten Seite
% headsepline: Trennt Textkörper von Headings durch Strich (entspr.: /footsepline)
% headinclude,footinclude: Kopf- und Fußzeile zählen zum Textkörper
% DIV=calc: Für die gewählten Optionen wird ein optimales Seitenverhältnis errechnet
% draft=true: Für Bilder wird die Box freigehalten, erheblicher Geschwindigkeitsvorteil.
% abstract: Setzt den Titel 'Zusammenfassung' vor den abstract

\usepackage{hyperref}

\usepackage{upgreek}
%\usepackage{subfigure}

%  %  %  %  Bindungskorrektour  %  %  %  %
% \KOMAoptions{BCOR=10mm}

%  %  %  %  Abkürzungen  %  %  %  %
% Das Einführen dieser Befehler verhindert Umbrüche bei mehrgliedrigen Abkürzungen
\usepackage{xspace}
\newcommand{\zB}{\mbox{z.\,B.}\xspace}
% Abkürzung für zum Beispiel


%  %  %  %  Einheiten  %  %  %  %
%\usepackage[thinspace,thinqspace,squaren,textstyle]{SIunits}
% Komfoatables Paket zum Einbinden von Einheiten


%  %  %  %  Kodierung, Schrift und Sprache  %  %  %  %
\usepackage[utf8]{inputenc}
\usepackage{palatino}
\usepackage[ngerman]{babel}
% damit man Text aus dem PDF korrekt rauskopieren kann


%  %  %  %  Grafiken, Tabellen, Mathematikumgebungen  %  %  %  %
\usepackage{graphicx}
\usepackage{tabularx}
\usepackage{xcolor}
\definecolor{halfgray}{gray}{0.55}
\usepackage{amsmath,amsfonts,amssymb}
\usepackage{flafter,afterpage}
\usepackage[section]{placeins}
\usepackage{setspace} \onehalfspacing
\usepackage[margin=8mm,font=small,labelfont=bf,format=plain]{caption}
\usepackage[margin=8mm,font=small,labelfont=bf,format=plain]{subcaption}

\numberwithin{equation}{chapter}
\numberwithin{figure}{chapter}
\numberwithin{table}{chapter}


%  %  %  %  Kopf- und Fußzeilen  %  %  %  %

% \renewcommand\frontmatter{\pagenumbering{Roman}}
\usepackage{chngcntr}
\counterwithout{footnote}{chapter}

% Zeilenabstand zwischen zwei Fußnoten:
\footnotesep 9pt
% Einrücken der Fußnoten:
\deffootnote[1.5em]{1em}{1.5em}{\thefootnotemark\ \ }
%
%\usepackage{fancyhdr}				% Paket für leicht konfigurierbare Kopf- und Fußzeilen
%\fancypagestyle{plain}{				% Neue Gestaltung der Chapter- Page
%\fancyhf{} 							% Clear all header and footer fields
%\renewcommand{\headrulewidth}{0pt}		% Keine Trennlinie zwischen Kopf- / Fußzeile und Textkörper
%\renewcommand{\footrulewidth}{0pt}}
%
%\fancypagestyle{myfoot}{				% Neue Gestaltung der frontmatter pages
%\fancyhf{}							% Clear all header and footer fields
%\fancyhead[RO]{\thepage}				% Seitenzahl außen auf ungeraden Seiten
%\fancyhead[LE]{\thepage}				% Seitenzahl außen auf geraden Seiten
%\renewcommand{\headrulerwidth}{0pt}	% Keine Trennlinie zwischen Kopf- / Fußzeile und Textkörper
%\renewcommand{\footrulerwidth}{0pt}}
%
%\pagestyle{fancy}					% Pagestyle fancy aktiviert selbstkonfigurierten Style
%\fancyhf{} 							% Alle Kopf- und Fußzeilenfelder werden zunächst bereinig
%\renewcommand{\headrulewidth}{0pt}		% Keine Trennlienie zwischen Kopfzeile und Textkörper
%
%%\renewcommand{\chaptermark}[1]{\markboth{#1}{}}
%%\renewcommand{\sectionmark}[1]{\markright{#1}{}}
%
%\fancyhead[RO]{\leftmark ~~~~ \thepage}
%\fancyhead[LE]{\thepage ~~~~ \nouppercase \rightmark}


%  %  %  %  Überschriften  %  %  %  %


%  %  %  %  Verzeichnisse  %  %  %  %

% % % Literaturverzeichnis % % %
%\usepackage{natbib}

% % % Inhaltsverzeichnis % % %
% Die Chaptereinträge:
\usepackage{titletoc}

\titlecontents{chapter}
	[0pc]
	{
		\addvspace{0.5pc}
		%\filouter}
	}
	{\sffamily\Large\thecontentslabel\quad\sffamily\Large}{}
	{\titlerule*[0.75pc]{}\enskip\rmfamily\Large \contentspage}  % Wäre mit Seitenzahl 																					rechtsbündig
	[\addvspace{.5pc}]

% Die Sectioneinträge:
\titlecontents{section}
	[3.78em]
	{}
	{\rmfamily\contentslabel{2.3em}\rmfamily}
	{\hspace*{-2.3em}}
	{\titlerule*[0.75pc]{.}\enskip\contentspage}
	[\addvspace{.1em}]

% Die Subsectioneinträge:
\titlecontents{subsection}
	[6.2em]
	{}
	{\rmfamily\contentslabel{2.3em}\rmfamily}
	{\hspace*{-2.3em}}
	{\titlerule*[0.75pc]{.}\enskip\contentspage}
	[\addvspace{.1em}]

%\titlecontents{subsection}
	%[6.8em]
	%{}
	%{\rmfamily\normalsize\contentslabel{3em}\rmfamily\large}
	%{\hspace*{-2.3em}}
	%{\titlerule*[0.75pc]{.}\enskip\contentspage}
\begin{document}

%  %  %  %  Titelseite  %  %  %  %
\begin{titlepage}
	\vspace*{\fill}

	\rule{\textwidth}{0.25pt}

	\vspace*{1cm}

	\begin{singlespace}
		\begin{center}	\Large	\bfseries
			Software Engineering 2
		\end{center}
	\end{singlespace}

	\vspace{1em}

	\begin{singlespace}
		\begin{center}	\Large \bfseries
		Pflichtenheft
		
		\vspace{2em}	\large
		Projekt:
		
		Entwicklung eines Software-Systems
		
		zur Simulation der Steuerung eines Liftes
		\end{center}
	\end{singlespace}

	\vspace*{5cm}

	\rule{\textwidth}{0.25pt}

	\vspace*{\fill}
\end{titlepage}

%  %  %  %  Inhaltsverzeichnis  %  %  %  %

\tableofcontents

%  %  %  %  Hauptteil  %  %  %  %
%% \mainmatter

\chapter{Einführung}
\paragraph{}
Im Rahmen der Belegarbeit im Fach \textsc{Software Engineering 2} ist ein Software-System zu implementieren, welches die Steuerung eines Liftes simuliert. Dieses Software-System soll in der Zukunft als Anschauungsmaterial im Lehrbetrieb verwendet werden. Studierenden soll damit ermöglicht werden, die Zusammenhänge zwischen real existierenden Automaten und der Thematik der Zustandsdiagramme zu erfahren.

\paragraph{}
In diesem Zusammenhang ergeben sich zusätzlich zu den Anforderungen an das Teilsystem \textit{\textbf{Liftsteuerung}}, spezielle Anforderungen an das Teilsystem \textit{\textbf{Visualisierung}} aus der Sicht des Lehrbetriebes. Im folgenden werden diese beiden Teilsysteme daher an verschiedenen Stellen getrennt voneinander betrachtet und beschrieben.

\paragraph{}
Das vorliegende Pflichtenheft dient der Beschreibung und Vereinbarung von Anforderungen an das Gesamtsystem und besitzt Vertragscharackter. Die darin enthaltenen Anforderungen wurden auf Basis von Kundengesprächen und Kundenvereinbarungen sowie der Aufgabenstellung formuliert\footnote{Meeting Minutes und Audiomitschnitte der beiden Kundengespräche sind unter folgendem Link zu finden: \url{http://goo.gl/UVHn2G}} . 

\chapter{Dokumentation der Anforderungen}
Anforderungen an ein Software-Produkt werden im Allgemeinen zunächst in funktionale und nicht-funktionale Anforderungen unterteilt. Erstere decken dabei die Fähigkeiten und die Beschaffenheiten ab, die der Benutzer der Software zur Problemlösung oder zur Erreichung seines Zieles benötigt. Nicht- funktionale Anforderungen unterteilen sich weiterhin in Rahmenbedingungen und Qualitätsanforderungen.

\paragraph{}
Im Folgenden werden die funktionalen Anforderungen an das Software-System aus den bereits angesprochen zwei Perspektiven betrachtet. Perspektive \textit{\textbf{A)}} bezieht sich auf das Teilsystem \textit{\textbf{Visualisierung}} und betrachtet es aus der Sicht des Benutzers. Diese Sicht wird im folgenden \textit{\textbf{Benutzersicht}} genannt. Unter Perspektive \textit{\textbf{B)}} wird das Teilsystem \textit{\textbf{Liftsteuerung}} aus der Sicht der Passagiere betrachtet. Diese Sicht wird im folgenden \textit{\textbf{Passagiersicht}} genannt.

\newpage
\section{Kontextdiagramm}
Um die Verschachtelung der beiden Perspektiven sowie die Schnittstellen des Systems zur Umwelt übersichtlich darzustellen eignet sich ein Kontextdiagramm. 

TODO: KONTEXTDIAGRAMM EINFÜGEN

\newpage
\section{Satzschablonen}
Im Folgenden werden die Anforderungen an das Software-System mit Hilfe von Satzschablonen\footnote{Eine Satzschablone ist ein Bauplan für die syntaktische Struktur einer einzelnen Anforderung. Der Einsatz der Satzschablone unterstützt den Autor einer Anforderung darin, die syntaktische Eindeutigkeit der Anforderung zu erreichen} dokumentiert.

\paragraph{}
Deren Auflistung unterscheidet dabei zwischen selbstständigen Systemaktivitäten\footnote{Diese Aktionen werden von der Liftsteuerung selbstständig ausgeführt. Über verschiedene Schnittstellen interagieren Benutzer, Passagier und Sensoren mit dem System.} und Benutzerinteraktionen\footnote{Über Benutzerinteraktionen kann der Benutzer des Software-Systems mit der Liftsteuerung interagieren.}. Weiterhin wird eine Nummerierung vorgenommen, welche in allen Teilen der Software-Dokumentation konsistent benutzt werden.

\paragraph{}
\textbf{Selbstständige Systemaktivitäten}
\begin{itemize}
	\item \textbf{ELV-001:} \newline
		Die Liftsimulation muss den Fahrstuhl nach oben fahren lassen.
	\item \textbf{ELV-002:} \newline
		Die Liftsimulation muss den Fahrstuhl nach unten fahren lassen.
	\item \textbf{ELV-003:} \newline
		Die Liftsimulation muss das Öffnen der Fahrstuhltür anzeigen.
	\item \textbf{ELV-004:} \newline
		Die Liftsimulation muss das Schließen der Fahrstuhltür anzeigen.
	\item \textbf{ELV-005:} \newline
		Die Liftsimulation sollte den aktuellen Zustand des Fahrstuhls anzeigen.
	\item \textbf{ELV-006:} \newline
		Die Liftsimulation sollte Zustandsübergänge des Fahrstuhls anzeigen.
	\item \textbf{ELV-007:} \newline
		Die Liftsimulation muss den Wechsel einer Etage anzeigen.
	\item \textbf{ELV-008:} \newline
		Die Liftsimulation sollte die Fahrstuhltür selbständig schließen, wenn 
		länger als 3 Sekunden keine Benutzerinteraktion durchgeführt wurde.
	\item \textbf{ELV-009:} \newline
		Die Liftsimulation muss eine Überlastung des Fahrstuhls durch zu 
		viele Passagiere anzeigen.
		\newline
		
	\item \textbf{ELV-010:} \newline
		Die Liftsimulation muss im Falle einer Überlastsituation\footnote{Eine Überlastsituation tritt ein, sobald mehr sich als 8 Passagiere im Fahrstuhl befinden.} in den Zustand \textit{\textbf{Überlast}} wechseln. Ausgehend von diesem Zustand  ist es dann nur noch möglich in den vorherigen Zustand zu wechseln, sofern die Überlastsituation durch das Verlassen von Passagieren aufgehoben wurde.
\end{itemize}

\paragraph{}
\textbf{Benutzerinteraktionen}
\begin{itemize}
	\item \textbf{ELV-011:}
		Die Liftsimulation muss dem Anwender die Möglichkeit bieten einen \textbf{\textit{Fahrtwunsch}}\footnote{Ein 						\textbf{\textit{Fahrtwunsch}} ist die Eingabe der Zieletage eines Passagiers über die innere Schaltfläche des Liftes.} für einen Passagier 		einzugeben.
	\item \textbf{ELV-012:}
		Die Liftsimulation muss dem Anwender die Möglichkeit bieten einen Passagier in den Fahrstuhl einsteigen zu lassen.
	\item \textbf{ELV-013:}
		Die Liftsimulation muss dem Anwender die Möglichkeit bieten einen Passagier aus dem Fahrstuhl aussteigen zu lassen.
	\item \textbf{ELV-014:}
		Die Liftsimulation muss dem Anwender die Möglichkeit geben, einen \textbf{\textit{Fahrstuhlruf}}\footnote{Ein \textbf{\textit{Ruf}} 		wird durch das Betätigen eines Etagenknopfes abgesetzt.} in jedem Stockwerk absetzen zu können.
	\item \textbf{ELV-015:}
		Die Liftsimulation sollte dem Anwender die Möglichkeit bieten eine \textbf{\textit{Vorrangschaltung}}\footnote{Der Benutzer kann einen 		\textit{Monteur} in den Lift einsteigen lassen, welcher die Möglichkeit besitzt einen \textit{Priorisierten Fahrtwunsch} einzugeben.} 			auswählen zu lassen.
\end{itemize}

\newpage
\section{Anwendungsfälle - Benutzersicht}
In dieser Sicht gibt es einen \textit{\guillemotleft \ abstrakten \ \guillemotright} Anwendungsfall \textit{\textbf{Passagier/Monteur steuern}}. Dieser gliedert sich in die unabhängigen Anwendungsfälle:

\begin{itemize}
	\item \textit{\textbf{Passagier-einsteigen}} (ELV-012)
	\item \textit{\textbf{Passagier-aussteigen}} (ELV-013)
\end{itemize}

TODO: DIAGRAMM EINFÜGEN

Die Eingangs- und Ausgangsdaten, sowie einige Bemerkungen zu den Anwendungsfällen sind in folgender Tabelle zusammengefasst.

 {
\vspace{1cm}
\hspace{-0,5cm}
\footnotesize
\begin{tabular}{|p{1,5cm}|p{2,5cm}|p{2,5cm}|p{2,5cm}|p{2cm}|}
	\hline
		\textbf{Funktion} &
		\textbf{Eingangsdaten} &
		\textbf{Ausgangsdaten} &
		\textbf{Bemerkungen} &
		\textbf{abstrakter AWD} \\
	\hline \hline
		\textit{Passagier \newline einsteigen} &
		Betätigung der entsprechenden Schaltfläche &
		Visuelle Bestätigung der Eingabe &
		Betreten mehr als 8 Personen den Lift kann eine Überlastsituation auftreten&
		\textbf{Passagier /\newline Monteur\newline steuern} \\
	\cline{1-4}
		\textit{Passagier \newline aussteigen} &
		Betätigung der entsprechenden Schaltfläche &
		Visuelle Bestätigung der Eingabe &
		&
		\\
	\hline
\end{tabular}
}

\section{Anwendungsfälle - Passagiersicht}
Aus dieser Perspektive ergeben sich drei voneinander unabhängige Anwendungsfälle:

\begin{itemize}
	\item \textit{\textbf{Fahrtwunsch - Passagier}} (ELV-011)
	\item \textit{\textbf{Fahrstuhlruf }} (ELV-014)
	\item \textit{\textbf{Fahrtwunsch - Monteur }} (ELV-015)
\end{itemize}

TODO: DIAGRAMM EINFÜGEN

Die Eingangs-, Ausgangsdaten, sowie Bemerkungen sind wiederum in folgender Tabelle zusammengefasst.

 {
\vspace{1cm}
\hspace{-0,5cm}
\footnotesize
\begin{tabular}{|p{2cm}|p{3cm}|p{3cm}|p{3cm}|}
	\hline
		\textbf{Funktion} &
		\textbf{Eingangsdaten} &
		\textbf{Ausgangsdaten} &
		\textbf{Bemerkungen} \\
	\hline \hline
		\textit{Fahrtwunsch \newline Passagier} &
		Etagenwahl (innen) &
		Leuchten der Wunschetage &
		Übergabe des Wunsches an Liftsteuerung \\
	\hline
		\textit{Fahrstuhlruf} &
		Drücken eines der beiden Rufknöpfe (außen) &
		Leuchten der Ruftaste &
		Übergabe des Wunsches an Liftsteuerung  \\
	\hline
		\textit{Fahrtwunsch \newline Monteur} &
		Drücken des Schlüsselsymbols und der Rufetage &
		Leuchten der des Schlüsselsymbols und der Wunschetage &
		Übergabe des Wunsches an Liftsteuerung  \\
	\hline
\end{tabular}
}

\section{Qualitätsanforderungen}
Benutzerfreundlichkeit und die intuitive Bedienbarkeit des Software-Systems haben vor allem im Lehrbetrieb große Wichtigkeit. Aktivitäten des Systems sollten erst nach der Interaktion des Benutzers beginnen und nicht automatisch starten. So ist sicher gestellt, dass der Benutzer in jeder Situation genügend Zeit hat, um das vergangene und zukünftige Verhalten des Systems nachvollziehen und durchdenken zu können. 

\paragraph{}
Weiterhin wird sichergestellt, dass jede Interaktion des Benutzers mit dem System eine Rückmeldung an den Benutzer gibt. Hier werden vor allem Methoden der visuellen Rückmeldung Anwendung finden.

\section{Rahmenbedingungen}

\chapter{Zustandsdiagramm des Liftes}

\chapter{Glossar}

\section{Allgemeiner Glossar}
\paragraph{Anwendungsfall}
Eine abstrakte Darstellung einer vom Software-System angebotenen Funktionalität (Aktivität). Er kapselt eine Menge von Aktionen, die sequentiell, bediengungsabhängig oder zyklisch abgearbeitet werden. Ein Anwendungsfall wird in Folge von Dateneingaben oder zeitlichen Ereignissen ausgelöst und führt in der Regel zu einem von au{\ss}en sichtbarem Ergebnis.

\paragraph{Anwendungsfalldiagramm}
Das Anwendungsfalldiagramm, kurz AWD, stellt die funktionalen Anforderungen (Aktivitäten) aus Sicht des Anwenders dar. Diese Aktivitäten werden zu den Beteiligten aus dem Kontext (Akteuren) in Beziehung gesetzt.

\paragraph{Akteur}
Ein Akteur ist die abstrakte Darstellung einer externen Instanz, die mit dem System kommuniziert.

\section{Projektbezogener Glossar}

\end{document}
